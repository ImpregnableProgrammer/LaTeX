\documentclass[11pt,twoside]{article}

\usepackage{amsmath}
\usepackage{mathtools}
\usepackage{amssymb}
\usepackage{fancybox}
\usepackage{fancyhdr}
\usepackage{graphicx}
\usepackage[T1]{fontenc}
% arithmetic for arguments to counter commands
% counters: https://www.overleaf.com/learn/latex/Counters
% \usepackage{calc}
\usepackage[left=1in,right=1in,top=1in,bottom=1in]{geometry}

% commands for rending number class symbols
\newcommand{\R}{\mathbb{R}}
\newcommand{\Z}{\mathbb{Z}}
\newcommand{\Q}{\mathbb{Q}}
\newcommand{\C}{\mathbb{C}}

% renders set of residue classes mod n (arg #1)
\newcounter{a}
\newcommand{\residues}[1]{
    \{
        \newcommand{\residue}[1]{
            \overline{\arabic{a}}
            \stepcounter{a}
            \ifnum\value{a}<#1
                , % output comma
                \residue{#1}
            \else
                \setcounter{a}{0}
            \fi
        }
        \residue{#1}
    \}
}

% commands for certain maths structures
\newcommand{\m}[1]{\begin{pmatrix}#1\end{pmatrix}}
\newcommand{\ovl}[1]{\overline{#1}}

\title{Daily Assignment 1}
\author{Rohan Kapur, Math 117 Summer 2023}
\date{\today}

\begin{document}

    \maketitle

    \begin{enumerate}
        \item Consider the set $\Z_7=\residues{7}$ of residue classes of integers mod 7.
        \begin{enumerate}
            \item Construct the multiplication table for the group $\left(\Z_7\backslash\residues{1},\cdot\right)$ where $\cdot$ is defined using representatives: $\ovl{m}\cdot\ovl{n}=\ovl{mn}.\\$
            \textbf{Answer:}

            \newcounter{row}
            \newcounter{col}
            \setcounter{row}{-1}
            \setcounter{col}{-1}

            \newcommand\TAB{&} % tab separator for columns
            \newcommand\NL{\\} % newline separator for rows

            % Calculate val = #1 (MOD #2)
            \newcounter{val}
            \newcommand{\MOD}[2]{\setcounter{val}{#1}\ifnum\value{val}<#2 \theval\setcounter{val}{0}\else\setcounter{val}{#1-#2}\MOD{\value{val}}{#2}\fi}

            % Calculate and output table (row, column) values
            % Spacing matters after output in the code!! Like after `$\ovl{\therow}$' there should be no whitespace in the code!
            \newcounter{prod}
            \newcommand{\cols}[1]{
                \ifnum\value{col}<0 
                    \ifnum\value{row}>-1 $\ovl{\therow}$\else$\cdot$\fi
                \else
                    \ifnum\value{row}<0 $\ovl{\thecol}$\else\setcounter{prod}{\value{row}*\value{col}}$\ovl{\MOD{\theprod}{#1}}$\fi
                \fi
                \stepcounter{col}
                \ifnum\value{col}<#1 \TAB \cols{#1}\else\setcounter{col}{-1}\fi
            }

            % Output correct row formatting and call `\cols' method for each row
            \newcommand{\rows}[1]{\cols{#1}\stepcounter{row}\ifnum\value{row}<#1\NL\hline\ifnum\value{row}=0\hline\hline\fi\rows{#1}\else\setcounter{row}{-1}\fi}

            % Render multiplication table with output given by `rows' for congruence classes modulo N
            \def\N{7}
            \begin{tabular}{c||c|c|c|c|c|c|c|}
                \rows{\N} \\
                \hline
            \end{tabular}
            \\
            \item Use part $(a)$ to find the multiplicative inverse of every nonzero element of $\Z_7$. \\
            \textbf{Answer:} $$ (\ovl{1})^{-1}=\ovl{1}, (\ovl{2})^{-1}=\ovl{4}, (\ovl{3})^{-1}=\ovl{5}, (\ovl{4})^{-1}=\ovl{2}, (\ovl{5})^{-1}=\ovl{3}, (\ovl{6})^{-1}=\ovl{6} $$

        \end{enumerate}

        \item Let $\mathbf{V}$ be a vector space over a field $\mathbf{F}$. Using only the definitions, prove the following for all $v \in V, a\in F$:
        \begin{enumerate}
            \item $0v=\mathbf{0}$ \\
            \textbf{Proof:} This is saying that $0v$ is equal to the additive identity $\mathbf{0}$ in $V$. To see this, notice that by distributivity of $F$ over $V$, $0v+0v=(0+0)v=0v$. Then by adding $-(0v)$ to both sides, we get $0v = \mathbf{0}$ as desired. $\blacksquare$
            \item $(-a)v=-(av)$ \\
            \textbf{Proof:} This is saying that $(-a)v$ equals the unique additive inverse $-(av)$ of $av$. To see this, notice that by distributivity of $V$ over $F$, $(-a)v+av=(-a+a)v=0v=\mathbf{0}$. Therefore $(-a)v$ is indeed the unique additive inverse of $av$ in $V$, so $(-a)v=-(av)$. $\blacksquare$
            \item $a\mathbf{0}=\mathbf{0}$ \\
            \textbf{Proof:} This is saying that $a\mathbf{0}$ is the additive identity in $V$. To see this, notice that by distributivity of $V$ over $F$, $a\mathbf{0} + a\mathbf{0}=a(\mathbf{0} + \mathbf{0})=a\mathbf{0}$. Then by adding $-(a\mathbf{0})$ to both sides, we get $a\mathbf{0}=\mathbf{0}$. $\blacksquare$
            \item If $av=\mathbf{0}$, then $a=0$ or $v=\mathbf{0}$. \\
            \textbf{Proof:} Suppose that $av=\mathbf{0}$, so $av$ is the unique additive identity in $V$. Now suppose that $v \ne \mathbf{0}$. 

        \end{enumerate}
        

    \end{enumerate}

\end{document}