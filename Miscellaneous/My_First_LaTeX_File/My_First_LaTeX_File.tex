% This is a comment in LaTeX!
% Use `pdflatex` to compile .tex files to LaTeX .pdf files
% Use `pdftex` for plain TeX files .tex to .pdf
% Use `kpsewhere` to find where various packages are located
% Use `tlmgr` to manage your Tex Live distribution
% Use `tex` or `latex` to convert from .tex to .dvi (https://en.wikipedia.org/wiki/Device_independent_file_format)
% Use `dvipdf` to convert from .dvi to .pdf
% Use `dvips` to convert from .dvi to .ps (PostScript)
% Use `ps2pdf` to convert from .ps to .pdf
% Use `texliveonfly` to automatically install missing packages for any .tex file
% No `man` or `info` pages for TeX utilities
% Instead, append `--help` to the end of any of them to get information about them
% /usr/bin/texlive/2016 is the TeX live main distribution directory
% All Tex Live utilities are located in /Library/TeX/texbin
% Tutorial: https://en.wikibooks.org/wiki/LaTeX
\documentclass[letterpaper,12pt]{article}
% Begin preamble
\title{\underline{My First \LaTeX{} Document!}}
\author{Rohan Kapur}
\usepackage{amsmath} % ``Import'' more advanced math typesetting tools
% Reduce margin sizes
\usepackage[margin=1in]{geometry}
% End preamble
\begin{document}
\maketitle
\begin{abstract}
  Well, this is apparently an abstract. Let us make it longer. Let us add a mathematical formula...How about the definition for fibonacci numbers?
  \[x_{0} = 0\]
  \[x_{1} = 1\]
  \[x_{n} = x_{n-1} + x_{n-2}\]
\end{abstract}
Well, here is my first ever LaTeX document! Some mathematical formulas inline: \(\frac{2}{3}e^{x+2}\) and on a new line!:
\[\frac{2}{3}e^{x+2}\]
Matrices!
\[
  \begin{bmatrix}
    3 & 4 & 5 \\
    5 & 6 & 7
  \end{bmatrix}
  \cdot
  \begin{bmatrix}
    3 & 4 \\
    5 & 6 \\
    6 & 7
  \end{bmatrix}
  =
  \begin{bmatrix}
    59 & 71  \\
    87 & 105
  \end{bmatrix}
\]
\end{document}
