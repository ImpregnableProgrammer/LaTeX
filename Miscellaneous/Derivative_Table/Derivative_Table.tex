% Differential Calculus
\documentclass[12pt]{article}
\title{\underline{Derivative Table}}
\usepackage[margin=1in]{geometry}
\usepackage{tasks} % For horizontal (rather than vertical) lists
\usepackage{datetime}
\linespread{1.6}
\newcommand{\Deriv}[2]{\(\displaystyle \frac{d}{dx} #1 = #2\)}
\begin{document}
\begin{flushright}
  Author: Rohan Kapur \\
  Created: \today
\end{flushright}
\centering{
  {\underline{\Huge{My Calculus Notes}}}
}
\begin{abstract}
  The Derivative of any continuous function \(f(x)\) is the function \(f^\prime(x)\) for which \(\forall{a}\) in the domain of \(f\), geometrically speaking, \(f^\prime(a)\) equals the slope of the line \textit{tangent} to the point \((a, f(a))\). In other words, \(f^\prime(x)\) is the instantaeous rate of change of \(f(x)\) at any point in its domain.
\end{abstract}
\begin{flushleft}{\subsection*{\underline{The Limit Definition of Derivatives}}}\end{flushleft}
\Deriv{f(x)}{\lim_{h \rightarrow 0}\frac{f(x+h)-f(x)}{h}} = \(\displaystyle \lim_{h \rightarrow a}\frac{f(h) - f(a)}{h - a}\)
\begin{flushleft}{\subsection*{\underline{Derivative Rules}}}\end{flushleft}
\vspace{-1.5cm}
%\begin{Large}
\begin{tasks}[label=$\bullet$](2)
  \task {\Deriv{x^{a}}{ax^{a-1}}}
  \task {\Deriv{f(x) \pm g(x) \pm \dots}{f^\prime(x) \pm g^\prime(x) \pm \dots}}
  \task {\Deriv{f(x)g(x)}{f^\prime(x)g(x)+f(x)g^\prime(x)}}
  \task {\Deriv{\frac{f(x)}{g(x)}}{\frac{f^\prime(x)g(x)-f(x)g^\prime(x)}{[g(x)]^2}}}
  \task {\Deriv{f(g(x))}{f^\prime(g(x))g^\prime(x)}}
  \task {\Deriv{f^{-1}(x)}{\frac{1}{f^\prime(f^{-1}(x))}}}
  \task {\Deriv{\sin(x)}{\cos(x)}}
  \task {\Deriv{\cos(x)}{-\sin{x}}}
  \task {\Deriv{e^x}{e^x}}
  \task {\Deriv{\ln(x)}{\frac{1}{x}}}
  \task {\Deriv{a^{x}}{a^{x}\ln(a)}}
\end{tasks}
%\end{Large}
\end{document}
