% +------------------------------------+
% |   Generated by www.docx2latex.com  |
% |   Version: 2.0.0                   |
% +------------------------------------+

\documentclass[11pt,twoside]{article}

\usepackage{amsmath}
\usepackage{amssymb}
\usepackage{fancyhdr}
\usepackage{float}
\usepackage[T1]{fontenc}
\usepackage{graphicx}
\usepackage[utf8]{inputenc}
\usepackage{mathtools}
\usepackage{wasysym}
\usepackage[paperheight=27.94cm,paperwidth=21.59cm,left=2.54cm,right=2.54cm,top=2.54cm,bottom=2.54cm]{geometry}

\setlength\parindent{0pt}

\newcommand{\m}[1]{\begin{pmatrix}#1\end{pmatrix}}
\renewcommand{\arraystretch}{1.3}\pagestyle{fancy}

\begin{document}

\title{Homework 2}
\date{July 30, 2023}
\author{Rohan Kapur, Math 111A Summer 2023}

\maketitle

\begin{enumerate}

	\item If true, show that a group is abelian if every subgroup is normal. Otherwise, give a counterexample.
	
	\textbf{Counterexample}: The quaternion group \( Q_8=\{1,-1,i,-i,j,-j,k,-k\}\) with \(i^2=j^2=k^2=-1,(-1)^2=1\) is an example of a \textit{non}abelian group whose subgroups are all normal.
 
	\item Suppose that \( N,M\) are two normal subgroups of \( G\) and that \( N\cap M = \{ e\}\). Show that for any \( n\in N, m\in M, nm = mn\).

	\textbf{Proof: }Since \( N,M\) are normal subgroups of \( G\), for any \( g\in G, gN = Ng\) and \( gM = Mg\). Then, since \( N\subset G\), we also have for any \( n\in N, nM = Mn\), or \( nMn^{-1} = M\), so for any \( m\in M, nmn^{-1}\in M\), and since \( M\) is a subgroup, we have \( m^{-1}\in M\), so by closure,\(\left(nmn^{-`1}\right)m^{-1}\in M\). Similarly, since \( M\subset G\), we have for any \( m\in M, mN = Nm\) or \( mNm^{-1} = N\), and since \( N\) is a subgroup, for any \( n\in N, n^{-1 }\in N\), so \( mn^{-1}m^{-1}\in N\), and by closure and associativity, \( n\left(mn^{-1}m^{-1}\right) =\left(nmn^{-1}\right)m^{-1}\in N\). Therefore \(\left(nmn^{-1}\right)m^{-1}\in N\cap M = \{ e\}\), so \(\left(nmn^{-1}\right)m^{-1} = e\), or \( nm = mn\). \textbf{\textit{QED}}

	\item If \( N\) is normal in \( G\) and \( a\in G\) is of order \( o(a)\), prove that the order, \( m\), of \( Na\) in \( G/N\) is a divisor of \( o(a)\).
	
	\textbf{Proof:} Let $ s = o(a) $. Then we have $ a^n  = e $, so $ Na^n = (Na^n) = Ne = N$. And since $ ord(Na) = m $, we also have $ (Na)^m = N $. Thus we must have $ m \le n $ since $ ord(Na) = m $. By the division algorithm, let $ n = qm + r $ for $q, r \in \mathbb{Z}$ with $0 \le r < n$. We need to show that $r=0$, so let $r = n-qm$ and suppose $r>0$. Then $a^r=a^{n-qm}=a^na^{-qm}=a^{-qm}$, which contradicts $r>0$. Therefore, we must have $r = 0$, and hence $m$ divides $a$. \textbf{\textit{QED}} 
	
	(I'm unsure if this is correct\dots)

	\item If \( G\) is a non-abelian group of order 6, prove that \( G\approx S_{6}\) (\( G\) is isomorphic to \( S_{6}\)).
	
	\textbf{Proof:} Not sure how to approach this\dots

	\item Let \( G_{1} \) be group of all nonzero complex numbers under multiplication, and let \( G_{2}\) be group of all real 2\( \times\)2 matrices of the form $ \m{a & b \\ -b & a \\} $
	where not both \( a\) and \( b\) are 0 under matrix multiplication. Show that \( G_{1}\) and \( G_{2}\) are isomorphic.
	
	\textbf{Answer: } Not sure\dots

\end{enumerate}
\end{document}